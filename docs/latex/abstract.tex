\newpage
\begin{center}
	\Large \textbf{Abstract}
\end{center}
\vspace{1cm}

\noindent
Time series forecasting plays a crucial role in various domains, enabling accurate predictions based on historical patterns. This project explores different time series machine learning techniques applied to four diverse datasets, each addressing a unique problem statement. The first dataset, Household Power Consumption, involves forecasting hourly and daily energy usage patterns to enhance energy efficiency. The second dataset, Predictive Maintenance, focuses on detecting potential machine failures using sensor data, contributing to proactive maintenance strategies. The third dataset, Walmart and Rossmann Store Sales, aims to predict retail sales while accounting for seasonal trends, promotional effects, and holidays. Lastly, the Historical Stock Market Data dataset is used to predict closing prices and identify potential trading signals, assisting in financial decision-making.  

\vspace{0.5cm}

\noindent
The project follows a structured approach, beginning with data wrangling techniques such as timestamp conversion, handling missing data, and resampling to maintain consistency. Feature engineering is employed to enhance predictive performance, incorporating moving averages, rolling windows, and external data sources like holidays, weather conditions, and macroeconomic indicators. A comprehensive model comparison is conducted, evaluating traditional statistical models (ARIMA/SARIMA), machine learning methods (Random Forest, XGBoost), and deep learning architectures (LSTM, GRU). Model performance is assessed using error metrics such as Mean Squared Error (MSE), Root Mean Squared Error (RMSE), Mean Absolute Error (MAE), and Mean Absolute Percentage Error (MAPE). Additionally, explainability techniques such as SHAP values and feature importance plots are applied to ensemble models, providing insights into the key drivers of predictions. Visualizations, including forecasted vs. actual values, seasonal trends, and anomaly detection, further illustrate the effectiveness of the chosen models. The results highlight the strengths and limitations of various approaches, demonstrating the applicability of machine learning and deep learning models in real-world time series forecasting problems.

\vfill
