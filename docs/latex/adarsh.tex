\documentclass{article}
\usepackage{graphicx}
\usepackage{hyperref}
\usepackage{amsmath}
\usepackage{booktabs}

\title{Walmart Recruiting - Store Sales Forecasting}
\author{}
\date{}

\begin{document}

\maketitle

\section{Introduction}
In today's fast-paced retail environment, accurate sales forecasting is crucial for businesses to make informed decisions about inventory management, pricing strategies, and resource allocation. This chapter delves into the tools and methodologies used in sales forecasting, focusing on the integration of data analysis and machine learning techniques. The discussion will highlight the importance of these tools in enhancing forecasting accuracy and provide insights into the output generated by these methods.

\section{Tools Used in Sales Forecasting}
\begin{enumerate}
    \item \textbf{Pandas and NumPy}: These libraries are foundational in data manipulation and analysis. Pandas is used for handling structured data, including tabular data such as spreadsheets and SQL tables, while NumPy provides support for large, multi-dimensional arrays and matrices, along with a large collection of high-level mathematical functions to operate on these arrays.
    
    \item \textbf{Matplotlib and Seaborn}: These visualization tools are essential for understanding data patterns and trends. Matplotlib is a comprehensive library for creating static, animated, and interactive visualizations in Python, while Seaborn extends Matplotlib's capabilities, providing a high-level interface for drawing attractive and informative statistical graphics.
    
    \item \textbf{ARIMA (AutoRegressive Integrated Moving Average)}: This is a popular statistical model for forecasting and analyzing time series data. It helps in understanding the patterns and trends in data over time, making it a powerful tool for predicting future sales.
    
    \item \textbf{XGBoost (Extreme Gradient Boosting)}: This is a machine learning algorithm that provides a fast and efficient way to build models using gradient boosting. It is highly effective in handling large datasets and is often used for regression tasks, including sales forecasting.
    
    \item \textbf{LSTM (Long Short-Term Memory) Networks}: These are a type of Recurrent Neural Network (RNN) well-suited for modeling temporal relationships in data. LSTMs are particularly useful for time series forecasting as they can learn long-term dependencies in data.
\end{enumerate}

\section{Data Overview}
The dataset used for this analysis includes several key components:

\begin{itemize}
    \item \textbf{Train Data}: This dataset contains historical sales data, including variables such as store ID, department ID, date, weekly sales, and whether it was a holiday. It provides the foundation for training forecasting models.
    
    \begin{figure}[h]
        \centering
        \includegraphics[width=0.8\linewidth]{Distribution weekly sales.png}
        \caption{Distribution of Weekly Sales}
        \label{fig:weekly_sales_dist}
    \end{figure}
    
    \item \textbf{Test Data}: This dataset is used to evaluate the performance of trained models by predicting sales for unseen data points.
    \item \textbf{Stores Data}: This includes information about each store, such as store type and size, which can influence sales patterns.
    
    \begin{figure}[h]
        \centering
        \includegraphics[width=0.8\linewidth]{store.png}
        \caption{Weekly Sales by Store}
        \label{fig:sales_by_store}
    \end{figure}
    
    \item \textbf{Features Data}: This dataset contains additional features that might impact sales, such as temperature, fuel prices, markdowns, CPI (Consumer Price Index), and unemployment rates.
\end{itemize}

\section{Output Analysis}
The output from these tools provides valuable insights into the sales forecasting process:

\begin{itemize}
    \item \textbf{ARIMA Model Output}: The Mean Squared Error (MSE) and Root Mean Squared Error (RMSE) values indicate how well the model fits the data. A lower MSE and RMSE suggest better model performance. However, ARIMA models may struggle with complex patterns or non-linear relationships in data.
    
    \begin{figure}[h]
        \centering
        \includegraphics[width=0.8\linewidth]{Screenshot 2025-03-24 225545.png}
        \caption{Weekly Sales over Time}
        \label{fig:sales_over_time}
    \end{figure}
    
    \item \textbf{XGBoost Model Output}: XGBoost typically offers better performance than ARIMA in terms of MSE and RMSE, especially when dealing with large datasets or complex interactions between variables. Its ability to handle non-linear relationships makes it a preferred choice for many forecasting tasks.
    \item \textbf{LSTM Model Output}: LSTMs can capture complex temporal dependencies, making them suitable for forecasting tasks where seasonality or trends are significant. However, they can be computationally intensive and may require careful tuning of hyperparameters.
    
    \begin{figure}[h]
        \centering
        \includegraphics[width=0.8\linewidth]{Screenshot 2025-03-24 225725.png}
        \caption{Sales Forecast for Store 1 and Department 1}
        \label{fig:sales_forecast}
    \end{figure}
\end{itemize}

\section{Detailed Explanation of Tools and Outputs}
\subsection{ARIMA Model}
ARIMA models are widely used for time series forecasting due to their simplicity and effectiveness in capturing trends and seasonality. However, they assume a linear relationship between past values and future predictions, which might not always hold true for complex datasets.

\begin{itemize}
    \item \textbf{Components of ARIMA}:
    \begin{itemize}
        \item AR (AutoRegressive): Uses past values to forecast future values.
        \item I (Integrated): Differencing to make the time series stationary.
        \item MA (Moving Average): Uses past errors as predictors.
    \end{itemize}
    \item \textbf{Limitations}: ARIMA models can struggle with non-linear relationships and may not perform well with datasets that have multiple seasonality or complex patterns.
\end{itemize}

\subsection{XGBoost Model}
XGBoost is a powerful machine learning algorithm known for its speed and performance. It is particularly useful for handling large datasets and can capture complex interactions between variables.

\begin{itemize}
    \item \textbf{Key Features}:
    \begin{itemize}
        \item Gradient Boosting: Combines multiple weak models to create a strong predictive model.
        \item Handling Missing Values: XGBoost can handle missing values directly, which is beneficial for datasets with incomplete information.
        \item Regularization: Helps prevent overfitting by adding penalties to large weights.
    \end{itemize}
    \item \textbf{Advantages}: XGBoost is highly efficient and can handle non-linear relationships, making it suitable for complex forecasting tasks.
\end{itemize}

\subsection{LSTM Model}
LSTMs are a type of RNN designed to handle the vanishing gradient problem in traditional RNNs. They are particularly effective for time series forecasting due to their ability to learn long-term dependencies.

\begin{itemize}
    \item \textbf{Key Features}:
    \begin{itemize}
        \item Memory Cells: Allow LSTMs to retain information over long sequences.
        \item Gates: Control the flow of information into and out of the memory cells.
    \end{itemize}
    \item \textbf{Advantages}: LSTMs can capture complex temporal patterns, making them suitable for forecasting tasks with strong seasonal components or trends.
\end{itemize}

\section{Humanized Perspective}
From a business perspective, accurate sales forecasting is not just about predicting numbers; it's about making strategic decisions that impact profitability and customer satisfaction. By leveraging these advanced tools, businesses can better anticipate demand fluctuations, optimize inventory levels, and tailor marketing strategies to maximize sales during peak periods.

Moreover, the integration of machine learning algorithms allows for the incorporation of external factors such as weather, economic indicators, and social trends, which can significantly influence consumer behavior. This holistic approach to forecasting ensures that businesses are well-prepared to respond to market changes and maintain a competitive edge.

\section{Discussion}
The choice of tool depends on the nature of the data and the complexity of the forecasting task. For simple time series data with clear trends and seasonality, ARIMA might suffice. However, for more complex datasets or when dealing with multiple variables, machine learning models like XGBoost or LSTM networks are more appropriate.

In scenarios where computational resources are limited, XGBoost might be preferred due to its efficiency and speed. On the other hand, if the dataset exhibits strong temporal dependencies, LSTMs could provide better insights into future sales patterns.

Ultimately, the selection of forecasting tools should be guided by the specific needs of the business and the characteristics of the available data. By combining these tools effectively, businesses can enhance their forecasting capabilities, leading to more informed decision-making and improved operational efficiency.

\end{document}
